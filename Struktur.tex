%%-------------------------------------------------------------------
\chapter{Struktur}
%%-------------------------------------------------------------------

%%-------------------------------------------------------------------
\section{Inhalt der Ausarbeitung}

Eine Ausarbeitung muss eine Motivation enthalten, aus welcher sich eine Klärung der Aufgabenstellung ergibt.
Diese selbst erarbeitete Aufgabenstellung wird wiedergegeben.
Die Grundlagen müssen derart skizziert werden, dass für den Leser die in der Arbeit durchgeführten Schritte, Prozesse und verwendeten Werkzeuge verständlich sind.

Die Literaturrecherche ist im Kontext der Klärung der Aufgabenstellung besonders wichtig, da anhand des aktuellen Stands der Forschung die in dieser Arbeit untersuchten Probleme ihre Motivation, ihren Hintergrund erlangen.

Die Durchführung der Arbeit wird mit dem notwendigen Detailgrad für das Verständnis der Umsetzung und das Verständnis der Evaluation des Problemlösungsansatzes beschrieben.

Die Evaluation bildet zusammen mit der Literaturrecherche den wichtigsten Teil der Diplomarbeit.
Das zugehörige Kapitel versucht, die Tauglichkeit des umgesetzten Lösungsansatzes für die gefundene Problemstellung zu beweisen durch eine Menge von Experimenten.
Jedes Experiment ist dabei selbst motiviert durch einzelne zu untersuchende Aspekte sowie das Gesamtverhalten und wird entsprechend detailliert beschrieben.


%%-------------------------------------------------------------------
\section {Aufbau der Ausarbeitung}

Die Strukturierung hängt von 1. individuellen Wünschen, 2. der Themenstellung und gegebenenfalls 3. dem Geschmack des jeweiligen betreuenden Mitarbeiters ab.

\subsection{Orientierung an vorhandenen Arbeiten}

Es ist in der Regel nicht sinnvoll, sich an vorhandenen Arbeiten zu orientieren, da jede Arbeit ihre eigene Strukturierung, ihren eigenen Aufbau benötigt.

\subsubsection{Ausnahme}

Gelegentlich kann es vorkommen, dass die selbständig ermittelte sinnvolle Struktur der vorhandener Arbeiten ähnelt. Das wichtige dabei ist jedoch, dass Ihre Struktur für Ihr Thema und Ihr Projekt passend ist.

\subsection{Wahl der Sprache}

Studentische Arbeiten dürfen mittlerweile sowohl in Englisch als auch in Deutsch verfasst werden, jedoch nicht gemischt.

Sollten Sie eine deutschsprachige Ausarbeitung verfassen, so beachten Sie bitte, dass Sie zu Beginn des Dokuments \verb|Dokument.tex| die deutsche Sprachvariante auswählen bei der Verwendung des Pakets \verb|cappstyle|.

Bei einer englischen Ausarbeitung ist entsprechend die englische Sprachvariante zu wählen über \verb|\usepackage[english]{cappstyle}|.
Zusätzlich ist dann zu beachten, dass zuoberst der Ausarbeitung eine deutschsprachige, etwa einseitige Zusammenfassung der Arbeit steht.
